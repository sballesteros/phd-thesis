
\section{Appendix 1: Intrinsic period of oscillation in a SIRS model}
\label{sec:appendix1}

Assuming a constant hosts population size, the $SIRS$ model is:

\begin{align}
  \frac{dS}{dt} & = -\beta S I + g (1-S-I)\\
  \frac{dI}{dt} & = \beta S I - \nu I\\
\end{align}

The $SIRS$ model can be simplified, resulting in a
reduction in the number of parameters. Measuring time
in units of duration of infection, $t'=t \nu$, we get the non dimensional system:

\begin{align*}
  \frac{dS}{dt'} & = -R_0 S I + e (1-S-I)\\
  \frac{dI}{dt'} & = R_0 S I - I\\
\end{align*}

With $R_0=\frac{\beta}{\nu}$ and $e=\frac{g}{\nu}$.

The model has two possible steady states
\begin{itemize}
\item Disease-free equilibrium: $S=1$, $I=0$
\item Endemic equilibrium: $S=\frac{1}{R_0}$, $I=\frac{e-e/R_0}{1+e}$.
\end{itemize}

The stability of the endemic equilibrium is determined by the
eigenvalues of the Jacobian matrix:
$$J=
\begin{pmatrix}
  -R_0 I -e & -R_0 S -e\\
  R_0 I & R_0 S-1 \\
\end{pmatrix}$$
Evaluated at the endemic equilibrium it results in:
$$\begin{pmatrix}
\frac{e(1-R_0)}{1+e} & -1-e \\
\frac{e(R_0 -1}{1+e} & 0 \\
\end{pmatrix}$$

The eigenvalues are solution of
\begin{align*}
\det{J-\lambda I_2} & = 0 \\
\lambda^2 - \frac{e(1-R_0)}{1+e} \lambda +e(R_0-1) & =0\\
\end{align*}

A straightforward calculation shows that the discriminant is negative and the eigenvalues thus complex:
$$\lambda_i = \frac{\frac{e(1-R_0)}{1+e} \pm i \sqrt{4 e (R_0-1)-\frac{e^2(R_0-1)^2}{(1+e)^2}}}{2} $$

The period of the dampened oscillations is given by 
\begin{align*}
  T & =2\pi \frac{1}{\Im{\lambda}} \\
& = 2\pi \sqrt{\frac{4}{4 e (R_0-1)-\frac{e^2(R_0-1)^2}{(1+e)^2}}} \\
\end{align*}

In the natural timescale, the period is divided by $\nu$
As $e \to 0$, the period can be simplified in :
$$T=2 \pi \sqrt{\frac{DL}{R_0-1}}$$ expressed in natural timescale with $D=1/\nu$ and $L=1/g$.


\begin{figure}[htb]
  \center
    \includegraphics[width= 0.6 \linewidth]{graph/endoT.eps}
    \caption{single subtype model with $\frac{1}{\nu}=2.66$ and 8 days$^-1$}
  \label{fig:intrinsicT}
\end{figure}

\clearpage
\newpage

\section{Appendix 2: Floquet}
\label{sec:appendix2}

We now consider the following non autonomous system:

\begin{align*}
  \frac{dS}{dt} & = -\beta(t) S I + g (1-S-I)\\
  \frac{dI}{dt} & = \beta(t) S I - \nu I\\
\end{align*}

with $\beta(t)=\beta_0(1+e \cos(2 \pi t))$.

For positive initial conditions and some parameters values, the seasonally forced $SIRS$ model admits a stable periodic solution of period T.
We will label limit cycle solutions by $\mathbf{\bar{x}}(t) = (\bar{S}(t), \bar{I}(t))$ in the following calculations and have $\mathbf{\bar{x}}(t + T ) = \mathbf{\bar{x}}(t)$ for all times, t. 
The curve, $\mathbf{\bar{x}}(t)$, cannot be calculated in closed form. However good estimates can be obtained via numerical integration of Eq. (1).

In order to study stability, we now consider a dynamical path beginning close to, but not on, the limit cycle, $\mathbf{\bar{x}}(t)$. If the limit cycle solution is stable then the difference between this path and the geometric curve of the limit cycle will decay as time progresses.  Similarly to the expansion about a fixed point, we can write this difference as $\epsilon \mathbf{\xi}(t) = \mathbf{x}(t) - \mathbf{\bar{x}}(t)$  where, again, $\epsilon$ expresses our anticipation that the deviation from the limit cycle is small. Expanding in powers of $\epsilon$ and letting $\epsilon \to 0$, one then finds that the time evolution of $\mathbf{\xi}(t)$ takes on the linear form, $\frac{d\mathbf{\xi}}{dt} = \mathbf{A}(t) \mathbf{\xi}$, where the matrix $\mathbf{A}(t)$ is the Jacobian matrix evaluated at the limit cycle. 
Therefore, due to the periodic nature of $\mathbf{\bar{x}}(t)$, all elements of $\mathbf{A}(t)$ are periodic.

An analytical tool to characterize the stability or otherwise of limit cycle solutions is Floquet theory, the mathematical theory of linear differential equations with periodic coefficients. Since, we have $\mathbf{A}(t+T)= \mathbf{A}(t)$, Floquet theory is applicable. In our case, T is the period of the mean-field limit cycle under consideration.
The general solution of
$$\frac{d\mathbf{\xi}}{dt} = \mathbf{A}(t) \mathbf{\xi}$$
takes the form:
$$\mathbf{\xi}(t) = \sum_{i=1}^n c_i e^{\mu_i t} \mathbf{p}_i(t)$$
$\mu_i$ are complex numbers called characteristic or Floquet exponents.
Floquet exponents can be calculated numerically by solving the matrix differential equation
$$\frac{d\mathbf{X}}{dt} = \mathbf{A}(t) \mathbf{X}$$
over one period (from $t=0$ to $t=T$) with the identity matrix as an initial condition ($\mathbf{X}(0)=\mathbf{I}$)
The matrix $\mathbf{X}(T)$ is known as a fundamental matrix.
Floquet multipliers, $\rho_i$ are the egeinvalues of $\mathbf{X}(T)$. 
$\rho_i$ are also the eigenvalues of the (linear) Poincaré map $x(t) \rightarrow x(t+T)$
From $\rho_i$, Floquet exponents, $\mu_i$ are calculated as $\mu_i= \frac{\ln(\rho_i)}{T}$

In term of Floquet multipliers, the general solution is
$$\mathbf{\xi}(t) = \sum_{i=1}^n c_i e^{\frac{\ln(\rho_i)}{T} t} \mathbf{p}_i(t)$$
expressing $\rho_i= |\rho_i|e^{i \arg(\rho_i)}$ we have:
$$\mathbf{\xi}(t) = \sum_{i=1}^n c_i e^{(\frac{\ln(|\rho_i|)}{T}+ i \frac{\arg(\rho_i)}{T}) t} \mathbf{p}_i(t)$$
that gives the damping rate and oscillation period

if $\Re(\rho_i)>0$ $\arg(\rho_i)= \arctan(\frac{\Im(\rho_i)}{\Re(\rho_i)})$
else $\arg(\rho_i)= \pi- \arctan(\frac{\Im(\rho_i)}{\Re(\rho_i)})$


%%% Local Variables: 
%%% mode: latex
%%% TeX-master: t
%%% End: 
