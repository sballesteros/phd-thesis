\documentclass{article}
\usepackage[utf8]{inputenc}
\usepackage[cyr]{aeguill}
\usepackage[francais]{babel}
\usepackage[authoryear]{natbib}

\usepackage{float, caption}
\usepackage{amsmath,amssymb,mathrsfs}


\usepackage{verbatim}
\sloppy


\title{Notes sur TDC}

\begin{document}
\maketitle

\section{\citet{Mantle1973}}
C'est l'épidémie en question:
\begin{itemize}
\item context général
\item symptomes des premières et deuxièmes infection
\item analyses sérologiques à 1 mois d'intervalle pour 2 patients dont un expat
\item hypothèses explicatives:
\begin{enumerate}
\item 2 agents infectieux introduit en même temps au début. Compte tenu de la nature explosive de l'épidémie et des doubles cas similaires on peut dire qu'ils devaient être très proches donc c'est absurde.
\item le Tristania a réintroduit un autre virus après 15 jours en mer. Peu probable dans la mesure où RAS sur le bateau.
\item Reinfection probable à cause: mutation antigenique; certains individus n'ont pas développer l'immunité après leur premier contact et il s'en est suivi soit de la réinfection intra-hôte soit de la réinfection par d'autres individus.
\end{enumerate}
\end{itemize}

\section{Taylor Robinson 1963 (demander à Bernard)}
Analyse sérologiques suite à l'épidémie de $H1N1$ de 1954 sur l'île (toute la pop est infectée):
\begin{itemize}
\item 1950: épidémie qui n'a touché que certaines personnes agées de plus de 30 ans.
\item analyse de 1955: tout le monde a des anticorps contre la grippe A. Les plus vieux ($>20$ ans) en ont contre des souches A plus vieilles et aussi contre la B.
\item analyse de 1961 (Cape Town): sur 19 sujets entre 20 et 25 ans seul 4 présentent des anticorps (en très faible quantité) contre H2N2 (1959). Toute la pop est vaccinée contre H2N2 et aussi contre la B (souche récente).
\item Depuis leur arrivée en Angleterre en 1961, 4 personnes agées sont mortes à cause d'autres maladies respiratoires. Pop très sensible aux infections respiratoires.
\end{itemize}

\section{\citet{McVernon2007}}
\begin{itemize}
\item en l'absence d'anticorps dus à de précedentes infections, la réponse immunitaire peut etre insuffisante pour conferer une protection même chez les adultes.
\item des infections répétées sur de courtes durées ont été reportées dans la naval apprentice school de Greenwich en 1924 (pas de données sérologiques) ainsi que lors des vagues (été, automne, hiver) en 1918-1919 en Angleterre.
\item d'habitude on observe 30\% d'asymptomatiques.
\item le $R_{0}$ est une valeur moyennée sur différents hôtes et situation environnementales. Par exemple dans un avion on a mesuré un $R_{0}$ de 40.
\item bien que les asymptomatic ne semblent pas être très infectieux on a relevé des cas de transmission en basse saison chez certaines familles.
\item dans les modèles les plus simples les personnes vaccinées sont considérées comme R mais il semble que les vaccins ne soient pas complétement protecteurs ce qui veut dire qu'il faut en tenir compte pour prévoir les seuils de couverture lors des campagnes de vaccination.
\item il y a des raisons de croire que les taux d'attaque de grippe sont plus contraint par la cross-immunité pré-existante et par les cas asymptomatic plutôt que par un petit $R_{0}$
\item les hypothèses et prédictions des modèles doivent être confrontées aux données disponibles.
\end{itemize}

\section{\citet{Tyrrell1967}}
analyse séro des hab avant et après leur voyage en Angleterre (vaccination):
\begin{itemize}
\item 1961: volcan actif; 264 hab évacué vers Cape Town par bateau puis vers l'Angleterre
\item avant \c ca sur l'île il y avait des common cold dès qu'un nouveau bateau arrivait. D'ailleurs dès qu'ils ont quitté leur iles beaucoup sont tombé malade avant d'arriver en Angleterre. Même après le vaccin contre H2N2, ils ont souffert de common cold durant les premiers mois en Angleterre et 3 vieux sont mort de pneumonie.
\item saignées réalisées: à leur arrivée à cape town/après la vaccination/à leur départ d'Angleterre (cf tableau). Pas de serum pour les enfants de moins de 8 ans. par contre opn a au moins la moitié des autres groupe d'age.
\item la réponse au vaccin est très bonne chez presque tout le monde et chez beaucoup d'individus il y a un rappel des anticorps de précédentes infections par la grippe. Ce constat touche toutes les classes d'age ce qui suggère que la pop a été confrontée à la même histoire d'infection ce qui est concordant avec le fait que les épidémies sont plutot récentes et que la plupart des gens y ont été infectés (1954).
\item les prélévements de 1963 montrent que les anticorps contre les précédentes souches ont diminué mais que par contre les anticorps contre les souches circulantes (celle du vaccin) ont augmenté, notamment chez les gens qui avait eu une faible réponse après le vaccin (infection naturelles ne rappellent pas les vieilles souches alors que la vaccin oui suite à quoi ces anticorps disparaissent assez vite).
\item pour les virus de parainfluenza on voit que cela touche plutot les adultes alors qu'en général ce sont plutôt les enfants qui sont touchés (comme pour la grippe de 1971). Les adultes réagissent comme des enfants qui auraient perdu leur protection maternelle.
\item relation entre le groupe sanguin et la susceptibilité (conclusion opposée à celle d'une étude précédente). Pour la réponse au vaccin pas de différence. 
\item la pop était homogène, c'est une expérience grandeur nature de la réponse à un vaccin. Il permet de confirmer l'hypothèse de Davenport comme quoi le fait que les vieux répondent moins bien au vaccin est dû à leur répertoire plutot qu'à l'effet de l'age: au plus on est vieux au plus on rappelle des anciennes souches et on développe moins d'aanticorps contre le vaccin.
\item un dosage avec des titre de 1/40 est considéré comme suffisant pour protéger contre la grippe A ou B (dans cette étude). 
\end{itemize}

\section{\citet{Shibli1971}}
description précise de la pop entre 1964 et 1968 ainsi que des 8 épidémies de rhume: 
\begin{itemize}
\item elles sont corrélées à l'arrivée de bateau en provenance de Cape Town (Tristania et GG). Les bateau venant d'ailleurs ne provoque pas d'épidémies. Ce serait du à la proximité de Cape Town (voyage de 6 à 8 jours) qui permet aux virus de se maintenir.
\item beaucoup de contacts dans la population, souvent des fetes avec tout le monde sous le même toit.
\item les expats sont ceux qui gèrent l'administration, l'ecole, le médecin avec leur famille (env 20 pers). 
\item quand quelqu'un vient voir le médecin on lui donne une fiche qu'il remplit tout au long de sa maladie. On en donne aussi à sa famille au cas où.
\item ces épidémies ont touché entre 40 et 100 hab à chaque fois (mais il y a surement du sous-reporting). Ce qui veut dire qu'en moyenne un hab subi moins d'un rhume par an ce qui est peu comparé à des communauté moins isolées.
\item même s'ils ont eu le temps en Angleterre de rencontrer beaucoup de rhumes il y a souvent des épidémies.
\end{itemize}

\section{\citet{Hammond1971}}
Fit un model $SIR$ avec une durée constante d'infectiosité sur 7 épidémies de rhume.
\begin{itemize}
\item ``The Tristanian community would in fact seem to approach very closely the ideal homogeneous and closed society whose theoretical consideration has provided a basis for epidemic theory since its inception.''
\item 2 épidémies de rhume en double vague: mai 1967 et fevrier 1968. Mais à chaque fois on a pu identifier que c'était à cause de l'arrivée d'un bateau avec des passagers infectés (\citet{Shibli1971})
\item le taux de report est d'environ 80\%
\item le R0 est compris entre 1.5 et 4 avec une valeur exterme à 6 (mais correspond à une deuxième vague)
\end{itemize}

\section{\cite{Brown1966}}
1964: épidémie de H2N2 sur l'archipel de YAP dans le pacifique. populations jamais exposées à la grippe:
\begin{itemize}
\item première épidémie sur YAP (3500 hab) au début de 1964
\item à partir de juin, l'épidémie atteind les petites iles (entre 11 et 600 hab) à coté (on pense que c'est à cause du bateau qui ramène les enfants sur les îles)
\item les dernières iles touchées l'ont été plus sévérement (en mortalité et en sévérité)
\item les taux d'attaque dans les pop isolée sont de l'ordre de 90-100\%
\item le taux de mortalité est principalement dû au manque de médicament et de médecin dans les petites iles (parfois personne ne peut s'occuper des malades)
\item les différence génétique sont impossibles à désentrelacer des différences de statut immunologique. L'augmentation de la virulence du virus est aussi avérée par le fait que la grippe tue des pop de plus en plus jeune au fur et à mesure qu'elle se propage.
\item etude séro avant épidémie: aucun anticorp sur 1 Ilafuk et Ulithi (à part 1 ou 2 enfants) et quelques anticorps contre H1N1 et Swine sur Woleai.
\item chez 4 individus de plus de 20 ans et ayant des anticorps contre H1N1 et Swine, l'infection par H2N2 a fait grimper de 4 ordre de grandeur le taux d'anticorps contre ces souches.
\item aucun anticorps contre le type B n'a été observé.
\item après infection par H2N2 1964 on observe sur 53 personnes (dépourvues d'anticorps contre la grippe): 90\% de cross-réactivité (en proportion d'individus) contre H2N2 1962; 75\% contre H2N2 1957; 23\% on une réponse (faible) contre H1N1 et Swine.
\end{itemize}

\section{\cite{Brown1969a}}
Sur des îles du pacifiques dont le statut immuno de la pop est décrit dans (\cite{Brown1966}), ils ont testé la réponse immunitaire après l'injection d'une et deux doses de vaccins contre H2N2 en 1965-1966:
\begin{itemize}
\item sur les îles étudiées la dernière épidémie de grippe remonte à 1920 (type A) ou 1952 (type A1). Pour les gens nait avant ces dates on trouve des anticorps contre A, A1 ou les deux.
\item 4 vaccins: H2N2 1964 (Taiwan, Sidney, Puerto Rico) 1962 (Japon)
\item on teste les sera contre: les 4 souches de H2N2, du A, du A1, du H2N2 1957 (Japon) et du H3N2(Aichi)
\item Chez les gens totalement naifs: les vaccins Taiwan et Sidney provoque des réponse homologues et hétéerologues alors que Puerto Rico et Japon produisent peu ou pas de réponse homo mais par contre réponse hétéro (bizarre)
\item Chez les gens qui ont déjà des anticorps contre A et/ou A1: pattern similaire au naif mais réponse quantitativement plus importante
\item Réponse post vaccin par H2N2 contre A et A1: seul les gens qui ont déjà rencontré A et/ou A1 ont une réponse contre ces virus après le vaccin. A noter que certain individu infecté seulement par A (A1) ont développé des anticorps contre A1 (A). A noter aussi que ce sont les individus vacciné par la souche la plus vieille (Japon 1962) qui ont eu la meilleure réponse contre A et A1. 
\item Comparaison premiere et deuxieme dose (boosting): 
\begin{enumerate}
\item chez 10 ado naifs: première réponse surtout homo (un peu hétéroH2N2 chez certain mais nettement inférieur à homo en quantité) deuxième réponse: tout le monde augmente sa réponse homo même chez ceux où pas de première réponse et tout le monde a une bonne réponse hétéro H2N2. Par contre aucune réponse hétéro A, A1. 
\item chez 4 ado plus vieux ayant déjà rencontré A1: la prémière dose de vaccin suffit à faire une réponse importante pour tous au niveau homo et au moins contre un hétéro H2N2. La deuxième injection n'aboutit pas des différence majeures par rapport aux naifs.
\item chez peu d'individus qui ont été infecté naturellement par H2N2 un an au paravant, grosse réponse après la première injection puis deuxième réponse plus faible lors du boosting.
\end{enumerate}
\item Comparaison entre réponse après infection naturelle et réponse après vaccination chez des sujets n'ayant jamais eu la grippe: qualitativement c'est pareil (homo et hétéro) mais quantitaivement la réponse post-infection et meilleure que post-vaccination. Nottamment, les anticorps contre A et A1 ne sont observés que post-infection.
\item aucun réponse contre H3N2, même vchez les individus qui avaient fortement répondu après le vaccin contre A et A1
\item lire le résumé page 333
\item cette étude a été validée par des test de HI assay (ici c'est de la neutralisation d'anticorps)
\item de manière générale, les individus qui ont déjà eu des infections répondent de manière plus hétéro. Les différences observées lors des différentes études résulye peut etre des différentes techniques utilisées. 
\item cette étude (tout comme celle de \citet{Tyrrell1967}) suggérent que l'original antigenic sin n'est prépondérant que lors d'infection répétée sur des temps court: pour nos pop peu exposées on voit que la réponse est bonne. 
\end{itemize}
 
 \section{\cite{Brown1969b}}
  suite de (\cite{Brown1969a}) où l'on voit l'effet d'un vaccin contre type B (2 doses en 1965-1966 et post-épidémie 1966):
  \begin{itemize}
\item Chez les sujets naifs, 80\% des gens n'ont pas réagi au vaccin après première dose (les autres très peu)
\item après le boost, ils ont une augmentation d'ordre 4 avec de la croos-immunité contre les autres souches de B (1966 mais pas 1940)
\item chez les sujets déjà infecté par B 15 ans plus tot, la réponse à la première dose équivault à la réponse au boost des naif. Par contre si on les boost la réponse est plus faible.
\item 5 mois après le boost, une épidémie de B chez les sujets boosté par vaccin contre B et A (témoins): aucun effet du vaccin si ce n'est contre les agravements (diarrhé, vomissement)
\end{itemize}

\section{\cite{Becker1983}}
étudie l'hypothese d'homogeneous mixing pendant les épidémies de common-cold entre 1964 et 1968 sur TdC (hypothèse utilisée par \citet{Hammond1971}):
\begin{itemize}
\item les habitants de TdC se répartissent en 7 groupes de contact: pas de groupe social;un des 5 groupes sociaux; habitent à la Station (expatriés). Certains de ces groupes s'overlapent.
\item bonne explication de l'hypothèse d'homogeneous mixing P.336.
\item cette hypothèse est très probablement vérifiée au sein des foyers ou dans les écoles mais pas dans une ville. Pour TdC on a tendance à supposer que c'est aussi vrai mais les données très précises sur chaque nouveau cas permettent de tester cette hypothèse statistiquement (basé sur l'idée que si l'hypothèse est vérifiée, n'importe quelle partition de la pop doit refléter une propagation de l'épidémie en rapport directe avec sa taille).
\item calcul sur les 255 habitants sans compter les expat car pas de donner précise sur leur sexe,age et hébergement.
\item les épidémies ont tendance a se propager plus rapidement au sein des foyers, il y a aussi un effet sexe et age sur la propagation (les enfants et les femmes sont plus touchés).
\item les auteurs concluent que s'il y a bien une hétérogénéité , rien ne permet de dire que les modèles s'en trouve perturbés par l'hypothèse d'homogeneous mixing.  
\end{itemize}

\section{\cite{FOX1982}}
suivi de 639 familles à Seattle entre 1975 et 1979:
\begin{itemize}
\item résultats préliminaire sur Seattle entre 1965 et 1969: lors de l'apparition de H3N2 en 1968 le type dominant pendant l'épidémie fut B (en taille et en durée). Des infections ont été identifiée en été pour A/H2N2 et B. Pour les deux types, les taux d'attaques les plus élevés se trouvent chez les 2-5 ans. La réinfection par la B est plus fréquente.
\item épidémies: 1975-1976: type B (suspecte à cause de la différence dans les réponse par HI et CF test); 1975-1976 et 1977-1978: type A/H3N2; 1978-1979: type A/H1N1. En 1976-1977: pas d'épidémie pendant l'hiver.
\item une augmentation de 4 ordre de grandeur pour les titrages est synomyme d'infection par l'antigen correspondant (car isolement du virus rendu difficile par la fréquence des prélévements). Parmi les infectés, chez 13\% (H3N2) 20\%(H1N1) et 27\% (B) des individus pas de réponse au 2 tests d'anticorps (problème dans la méthode ou mécanisme TdC?). Les uateurs en déduisent que environ 15\% de la pop a été infecté sans être identifiée.
\item avant chaque épidémie saisonnière il y a eu des ``herald wave'' au printemps précédent.
\item pour H3N2 on a un changement de cluster: 1975-1976: VI/75, printemps 1977: VI/75 et Tx/77 corcircule (herald wave), 1977-1978 épidémie de VI/75 et Tx/77 (corcircule)
\item hors saison on a quelques cas (moins de 10) de type B  et H3N2 chaque été et H1N1 à l'été 1978.
\item les épidémie ont touché en majorité les enfants: 5-9 ans pour H3N2, ados pour B et H1N1.
\item H1N1 a provoqué le plus grand pic (31\% de taux d'attaque) mais seulement 2\% chez les adultes. A comparer avec les 17-24\% de tauw d'attaque (6-13\% chez les adultes) pour H3N2 et B.
\item sur 14 ans (1965-1979) en suivant environ 65 familles, 271 personnes: 16\% (13\%) de taux d'attaque pour H3N2 (B). 3\% de reinfection pour les deux et 1\% de 3eme infection. Le taux d'attaque est inversement proportionnel à l'age. 
\item sur les 4 ans, cas  de réinfection par le même sous-type: 37 famille pour H3N2, 15 pour B et 13 pour H1N1. La réinfection est le plus souvent due au taux de HI (donc d'anticorps) qu'à de précendentes infection: 67-100\% de réinfection quand les titrages sont faibles. Parmi les jeunes (<20 ans) la maladie est développée autant après réinfection qu'après infection. Chez les adultes on observe une certaine resistance à la réinfection.
\item cite Stuart-Harris P.223: le taux d'attaque de l'année suivante peut être prédit par rapport au taux d'anticorps dans la pop après l'épidémie saisonnière. Résultat infirmés par les uateurs qui ne voit pas de relation entre le taux d'anticorps après l'épidémie de H3N2 en 1975-1976 (Vi/75) et la taille de l'épidémie en 1977-1978 (Tx/77 et Vi/75, proportion inconnue). Peut etre un biais du fait du changement de cluster. Par contre les auteurs montrent un accroissement du nombre d'individus avec un taux d'anticorps suffisant (>1:40) à l'automne entre 1975 et 1978 (46\%,59\%,63\%,72\%) chez les moins de 20 ans. Si on regarde l'accroissement de ce nombre on trouve : 1975-1976:13;1976-1977:4;1977-1978:9. C'est la trace du changement de cluster. Pour H1N1,, toujours pour les moins de 20 ans, on trouve: 26\% en 1977, 17\% en 1978 (après l'herald wave) et 48 \% après la gross épidémie de 1978.
\item pour l'épidémie de H3N2 en 1977-1978: lors de l'herald wave au printemps 1977, beaucoup d'individus présentaient un fort titrage contre Tx/77 alors que les titrages contre Vi/75 étaient plus faibles. Cela peut-il traduire une dominance de Tx/77 lors de l'herald wave ou du moins une forte réponse à ce nouveau cluster. Du coup il fut difficile de trouver les 4 ordre d'augmentation dans les titrages de ces individus lors de l'épidémie de 1977-1978 où Vi/75 et Tx/77 ont corcirculé. Soit les infections par Tx/77 n'ont pas été détectées, soit il n'y en pas eu beaucoup du fait de la protection (mais en même temps peu d'individus devaient être protégés par l'herald wave), dans tous les cas il semble que Vi/75 est été deux fois plus présente que Tx/77 en 1977-1978 (Hi results). Du coup les auteurs utilisent les résultats agrégés des deux clusters pour détecter de la reinfection. 
\item alors que les précedentes infection par type A semble conférer une protection contre reinfection, il semble que ce ne soit pas le cas pour la B (mais ce sont des conclusions à partir de l'étude de 1965-1969 donc pas top)
\item de toute évidence il y a un problème de protection chez les jeunes de moins de 20 ans (seuls ceux qui ont un taux d'anticorps >1:40 semble protégés contre la réinfection, bien que ceux qui sont réellement réinfecté développe tous la maladie comme pour TdC).
\item les auteurs supportent l'idée de persistance off-season de la grippe. 
\end{itemize}


\bibliographystyle{apalike}
\bibliography{/Users/tonton/Documents/Biblio/fichiers_bib/library}
\end{document}