
\chapter{Discussion}

\begin{quote}
  “I am sure that what any of us do, we will be criticized either for
  doing too much or for doing too little... If an epidemic does not
  occur, we will be glad. If it does, then I hope we can say... that we
  have done everything and made every preparation possible to do the
  best job within the limits of available scientific knowledge and
  administrative procedure.”

  —US Surgeon General Leroy Burney, Meeting of the Association of
  State and Territorial Health Officers, August 28, 1957
\end{quote}

\vspace{2cm}

Nous avons vu en introduction que la prise en compte de la variation
antigénique à l'échelle d'une infection humaine permettait
d'appréhender les différentes trajectoires évolutives ayant mené aux
principales maladies infectieuses humaines. Une des questions à
l'origine de ce travail concernait les conséquences populationnelles
de la variation antigénique en s'appuyant sur le cas des épidémies de
grippe A dans la population humaine. Nous avons vu qu'à cette échelle,
l'immunité de la population est le principal facteur qui structure
cette diversité antigénique et avons cherché à développer des modèles,
les plus simples possibles, pour pouvoir étudier de manière précise la
boucle de rétro-action entre immunité des populations, évolution
virale et dynamique épidémiologique.


\section{Résumé des principaux résultats}

Notre première étude s'est ainsi intéressée aux répercussions des
hypothèses effectuées pour décrire les systèmes multi-souches. Elle a
été largement motivée par le changement de paradigme concernant le
mode d'évolution des propriétés antigéniques de la grippe introduit
par l'article de \citet{Koelle2006} et le débat qui a suivi visant à
déterminer si cette évolution était ponctuée ou plus graduelle. D'une
manière remarquable, l'échappement ponctué à l'immunité de la
population induit des balayages sélectifs capables de rendre compte
des phylogénies observées pour HA1 ainsi que des dynamiques
épidémiologiques. De plus, dans ce contexte, une période d'immunité
temporaire totale agissant quelles que soient les souches ou les
sous-types rencontrés n'est pas nécessaire pour réduire la diversité
virale. D'un point de vue théorique, le changement est considérable
car on passe d'un système où la diversité virale est contrôlée par un
mécanisme de densité dépendance directe à un système où elle est
contrôlée par les dynamiques transitoires de remplacements. Nous avons
montré qu'en réalité cette différence était un peu moins marquée. En
particulier, nous avons mis en évidence que les dynamiques du modèle
utilisé par \citet{Koelle2006} (SBRI) différaient grandement de celles
produites par les autres modèles utilisés pour décrire les systèmes
multi-souches. Au coeur de cette divergence se trouve la combinaison
de l'hypothèse d'une immunité polarisée avec le postulat que
l'immunité croisée agit en réduisant totalement l'infectiosité. Cette
combinaison implique que des individus complètement immunisés pour une
souche donnée peuvent tout de même se faire réinfecter par celle-ci
(il ne deviennent néanmoins pas infectieux) et ainsi, qu'ils
bénéficient à chaque fois d'une chance d'acquérir une protection
totale pour les souches immunologiquement réactives avec la souche
rencontrée. Ce processus (absent dans tous les autres modèles étudiés)
induit une immunité croisée beaucoup plus prononcée et tempère alors
l'échappement à l'immunité ayant lieu lors d'un changement de cluster
antigénique. Cette immunité additionnelle acquise grâce aux
réinfections favorise le remplacement de l'ancien cluster résidant par
le cluster émergeant. Nous avons montré que ce processus de
réinfection avec extension du répertoire immunitaire lors d'exposition
aux souches auxquelles l'hôte est totalement immunisé (sans qu'il
devienne infectieux) est largement contradictoire avec le péché
antigénique originel (OAS). L'importance du péché antigénique originel
a par ailleurs été soulignée récemment dans un article de
\citet{Kim2009} mettant fin à la polémique soulevée par les travaux de
\citet{Wrammert2008}. Devant l'incapacité des autres modèles
n'incorporant pas ce processus à reproduire des dynamiques
transitoires mimant qualitativement des changements de clusters, nous
avons voulu savoir dans quelle mesure un modèle basé sur un scénario
de dérive antigénique graduelle permettait de décrire d'une manière
satisfaisante les données dont on dispose.

La théorie de l'échappement de l'immunité ponctuée offre un mécanisme
intuitif pour expliquer la variabilité des épidémies de grippe
observées dans la zone tempérée, les plus grands échappements à
l'immunité induisant de plus grandes épidémies. Étant donnés nos
résultats précèdents, nous avons cherché à déterminer dans quelle
mesure ce signal épidémiologique pouvait être un indicateur fiable des
variations ponctuelles d'échappement à l'immunité. Nous avons donc
regardé si le modèle le plus simple possible, mimant une dérive
antigénique graduelle pouvait générer des dynamiques similaires à
celles observées dans la zone tempérée. Cette étude a révélé que ce
modèle déterministe produisait des dynamiques chaotiques mais avec la
propriété remarquable d'être régulière. Ainsi notre modèle minimal
génère des épidémies de tailles différentes chaque année, avec une
légère variabilité dans la date de départ de l'épidémie, et ce avec un
échappement à l'immunité parfaitement graduel, chaque individu ayant
une probabilité constante par unité de temps de redevenir susceptible.
Le caractère chaotique de la dynamique illustre en outre que même dans
ce cas idéal, où la dérive antigénique est parfaitement prévisible, il
existe un niveau d'indétermination fondamentale à notre capacité à
prédire l'état du système. De manière intrigante, cet horizon de
prédictabilité (déterminé par l'exposant de Lyapunov dominant) est de
l'ordre de grandeur de 3 ans, durée moyenne de changement des clusters
antigéniques. Nous avons également montré que cette dynamique, dite
UPCA (pour Uniform Phase with Chaotic Amplitude), se produisait dans
de larges zones de paramètres réalistes pour la grippe, y compris pour
les paramètres que nous avons inférés à partir de deux jeux de données
et qu'elle était robuste à différentes formes de perturbations (y
compris la co-circulation de différents sous-types dans un contexte
spatial réaliste). Ce modèle pourrait être en mesure de fournir une
explication au constat que les saisons les plus sévères de grippe
saisonnière ne sont pas toujours associées à des nouveautés
antigéniques marquées et peuvent avoir lieu durant des années où un
même cluster antigénique reste présent. Les épidémies de 1989-1990 au
Royaume Uni ou de 1999-2000 aux USA en sont des illustrations
remarquables \citep{Viboud2006b} et restent à ce jour à notre
connaissance inexpliquées.

Ce résultat, nous a amené à revisiter la théorie de l'évolution
antigénique ponctuée de la grippe et à développer un formalisme
suffisamment tractable et général pour pouvoir être utilisé dans un
cadre d'inférence statistique. Pour ce faire nous avons utilisé
l'approche de \cite{Koelle2009} qui se focalise uniquement au niveau
du phénotype. Dans le contexte de la grippe, les clusters antigéniques
constituent alors un niveau d'abstraction privilégié. Toutefois, étant
donné l'importance de la dérive antigénique graduelle au sein d'un
cluster antigénique \citep{Shih2007, Russell2008}, nous l'avons
incorporée. Nous avons choisi de le faire d'une manière implicite,
évitant encore de décrire de nombreuses souches. Ces simplifications
nous ont permis de réduire considérablement le nombre de variables
d'état nécessaires pour définir le système et ainsi de nous affranchir
de la nécessité de devoir se limiter à des formalismes où le nombre
d'équations augmente linéairement avec le nombre de souches
(typiquement le modèle SBRI). A une autre échelle, ce formalisme
permet aussi de décrire la dynamique de sous-types de grippes et donc
d'étudier les conditions menant à leur remplacement durant les
pandémies. L'analyse de ce système nous a conduit à un résultat
surprenant. Il est en effet apparu impossible de pouvoir reproduire
des remplacements de clusters antigéniques pour les distances
antigéniques réalistes qui séparent ces clusters. Ce résultat pourrait
signifier que l'évolution antigénique de la grippe est plus graduelle
qu'on ne le pense car des échappements ponctués à l'immunité de la
population produisent des épidémies trop violentes suivies de trop
longues périodes réfractaires pour être comparable aux données.
L'étude de \citet{Russell2008} utilisant un nombre beaucoup plus élevé
de prélèvements que celle de \citet{Smith2004} (plus de 13000 face à
273) fournit quelques éléments de réponse allant dans le sens d'une
dérive antigénique essentiellement graduelle bien que la période de
leur étude (2002-2007) ne soit pas marquée par des changements de
clusters antigéniques conséquents. Ce résultat surprenant peut aussi
traduire l'importance des contraintes fonctionnelles, largement
reconnue que ce soit sur le plan théorique \citep{Belshaw2008} ou sur
le plan empirique \citep{Rambaut2008}. Ces contraintes pourraient
compenser le gain de fitness relative acquis par un nouveau cluster
antigénique échappant à l'immunité de la population et ainsi éviter de
trop grandes épidémies. Nous avons aussi considéré d'autres hypothèses
et montré qu'une période temporaire d'immunité totale (classe $Q$)
pouvait améliorer la capacité de notre modèle à reproduire les
données, particulièrement si celle-ci est associée à un mécanisme de
``boosting'' immunitaire et des $R_0$ suffisant élevé pour que ce
``boosting'' ait un effet considérable. Dans ces conditions, les
individus partiellement protégés chez qui la protection partielle
suffit à éviter l'infection se retrouvent quand même dans la classe
$Q$. D'une manière plus classique, tous les individus infectés se
retrouvent temporairement dans cette classe après guérison. Ce
processus permet ainsi de réduire le nombre de susceptibles
disponibles pour les nouveaux variants antigéniques et tempère alors
l'intensité de l'épidémie faisant suite à un changement de cluster. La
classe $Q$ avec mécanisme de ``boosting'' rend donc les remplacements
possibles de deux façons : en augmentant l'interaction compétitive
entre les souches mais surtout, en tempérant les épidémies. Introduit
sous cette forme, la classe $Q$ n'est pas si différente du processus
``caché'' induit par le modèle $SBRI$ mis en évidence au chapitre 2.
Une différence de taille existe cependant entre ces 2 façons de
``booster'' l'immunité: la classe $Q$ ne la ``boost'' que
temporairement et sans mise à jour du répertoire immunitaire des hôtes
ne succombant pas à l'infection, tandis que le modèle SBRI induit des
extensions du répertoire immunitaire définitives lors d'expositions
aux souches auxquelles l'hôte est totalement immunisé sans qu'il
devienne infectieux, en désaccord avec l'OAS. Par ailleurs, là où le
``boosting'' est systématique avec la classe $Q$, il devient dépendant
de la distance antigénique pour le modèle $SBRI$. Si la mise en
évidence de ces mécanismes de ``boosting immunitaire'' constitue un
résultat intéressant, on ne peut toutefois pas exclure un rôle
important de l'hétérogénéité spatiale pour tempérer l'intensité des
épidémies.

%correlation R0 derive antigénique. avertissement état immunitaire peut
%varier d'une pop à une autre viboud smoldering pand. peut induire des
%biais considérable dans l'estimation de R0.



%calculer l'effet de la classe $Q$.



% abstract papier 3: finally, we show that both koelle and fergusion
% theory share an unpreviously recognised point by relying on the
% decisive assumption of cross immune boosting, a process during which
% already immunized hosts can gain additional cross protection without
% becoming infectious following additional exposure.

\section{Perspectives}

Le formalisme développé dans le chapitre 4 est suffisamment tractable
pour être couplé à une approche d'inférence statistique et de
sélection de modèles. Nous pensons donc qu'il permettra de mettre en
avant les processus clés impliqués dans la phylodynamique de la
grippe. En particulier, nous espérons pouvoir déterminer dans quelle
mesure les processus complexes de ``boosting'' immunitaires sont
nécessaires en présence d'une hétérogénéité spatiale locale accrue.
Cela nous semble particulièrement important étant donné que, dans le
contexte de la théorie de l'échappement à l'immunité ponctué, le rôle
primordial des processus de ``boosting'' immunitaire (classe $Q$,
modèle SBRI), semblent être de tempérer les épidémies. En effet, les
modèles sans ``boosting'' immunitaire induisent suffisamment de
compétition entre les clusters antigéniques pour provoquer des
balayages sélectifs lors de l'apparition de nouveaux clusters, mais
ils produisent alors des épidémies trop violentes suivies de trop
longues périodes réfractaires pour être réalistes. A ce titre, les
approches visant à prendre en compte simplement l'hétérogénéité
spatiale à une échelle locale sous la forme d'exposants sur les termes
de la loi d'action de masse sont un outil précieux \citep{Liu1987,
  Roy2006}. De plus amples travaux sont nécessaires dans ce domaine
pour mieux comprendre le rôle de ces exposants et déterminer s'ils
sont à même de mimer implicitement l'effet des réseaux de contacts
complexes de la population humaine.

En attendant ces confirmations, nos résultats révèlent que les
processus d'acquisition de l'immunité à l'echelle individuelle peuvent
avoir une importance considérable à l'échelle de la population. En
particulier il nous apparaît essentiel que les possibilités d'un
``boosting'' immunitaire régi par la classe $Q$ soient clarifiées
empiriquement. Notre travail suggère spécialement la nécessité d'une
meilleure compréhension de la façon dont interagissent les mécanismes
du péché antigénique originel (OAS) et les processus immunitaires
aboutissant à la mise en place d'une immunité cellulaire conférant une
protection totale hétéro-sous-typique mais limitée dans le temps. Si
ces 2 processus sont bien reconnus, que ce soit chez les modèles
animaux \citep{Grebe2008, Kim2009} ou lors d'études épidémiologiques
chez l'homme \citep{Slepushkin1959, Epstein2006}, les mécanismes
impliqués pour leur élicitation restent en partie à découvrir.

L'étude de \citet{Kim2009} offre une perspective intéressante à ce
sujet car les auteurs proposent un mécanisme pour l'OAS. Leur étude
suggère en effet que l'OAS pourrait être dû à une compétition entre
les lymphocytes B mémoires partiellement réactifs et les lymphocytes B
naïfs. Ceci se comprend en considérant une infection séquentielle par
deux souches ($s1$ et $s2$) immunologiquement partiellement réactives.
Comme le révèle l'étude de \citet{Kim2009}, la première infection
génère des anticorps spécifiques à $s1$ seulement, ainsi que des
anticorps partiellement réactifs à $s2$. Cependant, la situation est
différente lors de la seconde infection par la souche $s2$ car à
présent, les anticorps partiellement réactifs envers $s2$, générés
lors de l'infection par $s1$, dominent la réponse immunitaire et
pénalisent le déclenchement d'une nouvelle réponse spécifique à $s2$.
Ce phénomène pourrait s'expliquer par le fait que l'hemagglutinine
permet un accrochage des virus à l'ensemble des lymphocytes B quel que
soit leur récepteur caractéristique (BCR) via l'acide sialique. Cela
pourrait favoriser la capacité des lymphocytes B à présenter les
antigènes spécifiques des virus entrés en contact avec eux et ainsi
biaiser la présentation des antigènes spécifiques en faveur des
lymphocytes B, au détriment des cellules dendritiques. Ce mécanisme
pourrait conduire à des signaux d'activation sub-optimaux favorisant
la réponse immunitaire médiée par les lymphocytes B mémoires seulement
partiellement adaptés à $s2$ par rapport à l'activation de lymphocytes
B naïfs. Dans ce contexte, l'infection par $s2$ conduira à une charge
virale considérable au sein de l'hôte, bien que réduite par rapport à
celle ayant lieu dans un hôte naïf \citep{Kim2009}.

Au vu de ce mécanisme, dans le cas où l'OAS favorise une infection, il
est réaliste de considérer que la composante cellulaire de la réponse
immunitaire sera déclenchée pour éliminer le virus, conférant ainsi
une protection totale de courte durée \citep{Grebe2008}. La
combinaison de l'OAS et de la classe $Q$ sans ``boosting''
immunitaire, se comprend donc relativement bien lorsqu'une exposition
par une souche proche se traduit par une infection. La situation est
plus délicate dans le cas du ``boosting'' immunitaire ayant lieu
lorsque l'exposition ne conduit pas à l'infection. Dans ce dernier
cas, étant donné que l'hôte ne sera pas infectieux, on sait que sa
charge virale sera relativement faible voire nulle si les anticorps
présents chez l'hôte parviennent à neutraliser les virus très tôt. On
peut donc se demander dans quelle mesure une telle infection est
capable de déclencher une réponse immunitaire cellulaire. S'il a été
montré qu'une vaccination pouvait suffire à déclencher la classe $Q$
\citep{Slepushkin1959}, il est difficile de savoir dans quelle mesure
une exposition n'engendrant pas d'infection peut être comparable à une
vaccination. Lorsque l'OAS s'applique, peut-être que la charge virale
est suffisamment élevée pour tout de même déclencher la classe $Q$,
mais dans ce cas, l'hôte peut-il toujours être considéré non
infectieux ? Une description plus fine entre charge virale et niveau
d'infectiosité serait particulièrement précieuse dans ce contexte.
Nous avons vu en introduction que l'emploi de modèles intra-hôtes
relativement simples pouvait permettre de décrire cette relation avec
un rôle important de la fonction de transmission \citep{King2009}. La
présence de nombreux individus asymptomatiques est en outre une
incitation de plus pour regarder plus précisement dans cette direction
\citep{Carrat2008}.


Nous avons par ailleurs mis en évidence (annexe D) que les processus
d'acquisition de la réponse immunitaire pouvaient être fortement
variables selon que les hôtes soient naïfs ou pas. Pour se faire, nous
nous sommes intéressés à la première apparition de H3N2 sur l'île
isolée de Tristan da Cunha. Cette population où H3N2 est apparu
seulement en 1971, soit 3 ans après la pandémie de ce sous-type,
présente la particularité remarquable d'être exceptionnellement
susceptible à la grippe, principalement en raison du très faible
nombre d'expositions des habitants de cette île isolée. De façon tout
à fait surprenante, l'introduction de H3N2 dans cette île a produit
deux épidémies successives de grippe (confirmée par des analyses
sérologiques) séparées de moins d'un mois. Parmi les 284 habitants,
96\% ont subi au moins une infection et 32\% ont été infectés à deux
reprises, l'essentiel des réinfections ayant lieu lors de la seconde
vague. Nous avons cherché à déterminer quels processus étaient à même
de rendre compte de cette situation. Nous avons adopté une approche de
comparaison de modèles en mettant en contraste trois processus :
\begin{itemize}
\item La nécessité d'infections multiples avant de développer une
  immunité protective de longue duré (comme mise en évidence par
  \citet{Mathews2007}).
\item Une protection seulement partielle après la première infection.
\item Une mutation permettant au virus d'échapper à l'immunité
  contractée lors de la première vague.
\end{itemize}

Les résultats de cette analyse, décrits en détail en annexe D, nous
ont permis de valider la première de ces hypothèses. Ainsi, ces deux
vagues pandémiques sont en accord avec un modèle où seulement 50\% des
individus développent une immunité de longue durée après la première
attaque.  Il est difficile de savoir dans quelle mesure ce résultat
est généralisable, cependant, nous pensons qu'un processus similaire
pourrait avoir lieu de manière courante chez les individus
immunologiquement naïfs et donc typiquement chez les enfants.

Ce résultat incite à prendre en considération le rôle des infections
répétées. Si nous avons beaucoup parlé de distance antigénique dans ce
travail de thèse, il nous semble important de clarifier par la suite
le rôle du nombre d'infections dans l'acquisition d'une réponse
immunitaire de plus en plus protégeante et robuste à la dérive
antigénique virale. La réponse immunitaire cellulaire, dirigée contre
des protéines conservées à un rôle considérable dans ce domaine et les
infections multiples, en stimulant de plus en plus cette immunité,
pourraient peut-être à terme la rendre plus persistante, contrecarrant
ainsi l'échappement viral dû à la dérive antigénique (et favorisé par
l'OAS favorisant à son tour les infections répété et donc la
possibilité d'augmenter une réponse immunitaire fortement
réactive...). La suggestion que des infections naturelles (par
opposition au vaccin) répétées par la grippe dans le jeune âge
permettent une meilleure protection contre la grippe une fois âgé (en
plus de poser des questions passionnantes sur les politiques
vaccinales à adopter) incite fortement à regarder de plus près ces
questions \citep{Carrat2006a}.


\vspace{2cm} 

Une description fine des processus immunologiques peut avoir des
répercussions importantes à l'échelle de la population. La complexité
des modèles prenant en compte l'interaction de différents types
antigéniques à l'échelle de la population rend pour le moment
cependant difficile leur couplage avec des modèles décrivant la
relation entre le système immunitaire et la population virale au
niveau intra-hôte. A terme on peut toutefois espérer parvenir à
trouver un formalisme adapté à cette tâche, peut-être sous la forme
intermédiaire entre modèles individus centrés et systèmes dynamiques.
Un tel cadre unifié nous permettrait alors d'appréhender de façon
globale les interactions complexes et multi-échelles entre processus
immunologiques, épidémiologiques et évolutifs.


%A terme, o
%
%
% Les études de
%\citet{Lange2009}, \citet{Read2006} et \citet{King2009} illustrent
%particulièrement bien comment la combinaison de processus
%immunologiques à l'echelle intra-hote, couplé avec des processus
%épidémiologiques ainsi que des contraintes physoliogique permet de
%rendre compte des rétroactions entre dynamique épidémiologique et
%évolutive.
%
%importance du seuil intra hote pour invalider le SBRI...
%
%Il est admis que les anticorps préexistant jouent un rôle clef dans la
%protection contre les infections grippales. Cependant,
%\citet{Wrammert2008} ont montré récemment que les anticorps produit de
%novo par les cellules sécrétrices d'anticorps (ASC) relativement tôt
%suite à une vaccination (pique au 7eme jour) pouvait aussi avoir un
%rôle important en produisant une grande quantité d'anticorps
%spécifique aux antigénes présenté et ce apparemment sans être affecté
%par le péché antigénique originel (les anticorps produits ayant autant
%d'affinité si ce n'est plus avec les souches du vaccin actuel qu'avec
%celles des vaccins précédents. D'une manière remarquable les ASC ayant
%produit ces anticorps ont accumulé plus de mutations somatiques que
%n'importe quel population normal de lymphocytes B suggérant que ces
%cellules proviennent de lymphocyte B ``mémoires'' ayant accumulé des
%mutations sur les précédents cycles d'activations
%\citep{Wrammert2008}. Indépendamment des perspectives intéressantes de
%thérapie fournie par cette découverte cela peut fournir un argument
%allant dans le sens d'une réponse immunitaire multiplicative.
%
%
%
%discussion de la classe Q
%
%\cite{Slepushkin1959}
%\cite{Epstein2006}
%\cite{Carrat2006}
%\cite{Grebe2008}





%Comme l'a écrit Michel Tibayrenc dans l'éditorial du premier numéro du
%journal Infection, Genetics and Evolution nous vivons donc ``l'age
%d'or de la génétique mais l'age sombre des maladies
%infectieuses''. Cet ``age d'or de la génétique'' s'accompagne d'une
%augmentation sans précédent des capacités de calculs, plus que jamais
%nécessaire pour pouvoir analyser les données complexes fourni par le
%sequencages d'isolat provenant du monde entier \citep{Holmes2007}.
%Cette situation offre l'opportunité jamais atteinte auparavant de
%pouvoir pouvoir étudier de l'échelle intra-individuel à planétaire
%comment émerge la boucle de retro-action éco-évolutive façonnant les
%maladies infectieuses.



%%% Local Variables: 
%%% mode: latex
%%% TeX-master: "../phD"
%%% End: 
