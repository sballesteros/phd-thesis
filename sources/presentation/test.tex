\documentclass{beamer}

\mode<presentation> {
   \usetheme{Darmstadt}
  % \usetheme{Berlin}
  % \usetheme{Warsaw}
  % \usetheme{Singapore}
  % ou autre ...

  \setbeamercovered{transparent}
  % ou autre chose (il est également possible de supprimer cette ligne)
}

\usepackage[english]{babel}
% or autre comme par exemple \usepackage[english]{babel}
\usepackage{pstricks,pst-grad}
\usepackage{multicol}
\usepackage[utf8]{inputenc}
% or autre
\usepackage{exscale}
%\usepackage{authblk}
\usepackage{soul}
\usepackage[normalem]{ulem}
\usepackage{times}
\usepackage[T1]{fontenc}
\usepackage{natbib}
\usepackage{verbatim}
\usepackage{graphicx}
\usepackage{array, multirow}
\usepackage{color}
\usepackage{amsmath}
\usepackage{verbatim}
\usepackage{pgf}
\usepackage{tikz}
\usetikzlibrary{arrows,snakes,backgrounds,shapes}


\title[Titre court] % (facultatif, à utiliser uniquement si le titre de l'article est trop long)
{Ecology and Evolution of acute infectious diseases: the case of influenza}
%\subtitle {}

\author{Sébastien Ballesteros}

\institute[UMR 7625 Ecologie Evolution]
{
  UMR 7625 Ecologie Evolution\\
  Équipe Eco-Evolution mathématique\\
  ENS Ulm, UPMC
}

\date[Version courte] % (facultatif)
{18/12/2009 / Paris}

% À supprimer si vous ne voulez pas que la table des matières apparaisse
% au début de chaque sous-section :
% 
\AtBeginSubsection[] {
  \begin{frame}<beamer>{Contents}
  \begin{footnotesize}
    \tableofcontents[currentsection,currentsubsection]
  \end{footnotesize}
  \end{frame}
}

% Si vous souhaitez recouvrir vos transparents un à un,
% utilisez la commande suivante (pour plus d'info, voir la page 74 du manuel
% d'utilisation de Beamer (version 3.06) par Till Tantau) :

% \beamerdefaultoverlayspecification{<+->}


\begin{document}

\begin{frame}[fragile]
  \frametitle{Reproducing influenza phylodynamics}

%\only<1>{
\begin{verbatim}
t0  = [00000000000000]
t9  = [01000000000000]
\end{verbatim}
%}

%\only<2>{
%\begin{verbatim}
%t0  = [00000000000000]
%t9  = [01000000000000]
%\end{verbatim}
%}
%
%\only<3->{
%\begin{verbatim}
%t0  = [00000000000000]   t31 = [01000000000001]
%t9  = [01000000000000]   t47 = [00001000000000]
%t22 = [01000000010000]   t59 = [10000000010000]
%\end{verbatim}
%}

  \begin{columns}
    \begin{column}{0.35 \linewidth}
      \includegraphics<1->[height=1 \textwidth]{graph/ferg.png}

\only<1->{
      \begin{tiny}
        Ferguson \textit{et al.} (2003)
      \end{tiny}}
    \end{column}
    \begin{column}{0.65 \linewidth}
      \begin{block}<4->{Results}
        \alert{Explosive diversity of strains} and
        unrealistically high incidence
      \end{block}
      \begin{alertblock}<5->{Key theoretical question}
        How to restrict strains diversity ?
      \end{alertblock}
    \end{column}
  \end{columns}
  
\end{frame}

\end{document}
%%% Local Variables: 
%%% mode: latex
%%% TeX-master: t
%%% End: 
